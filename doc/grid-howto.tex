\documentclass[11pt,a4paper,headinclude,footinclude,DIV14,BCOR8.25mm,titlepage,twoside,openright,normalheadings]{scrreprt}
\usepackage[automark]{scrpage2}
\usepackage[ansinew]{inputenc}
%\usepackage{german}
%\usepackage{bibgerm}
\usepackage{amsmath}
\usepackage{amsfonts}
\usepackage{theorem}
\usepackage{color}
\usepackage{listings}
\lstset{language=C++, basicstyle=\ttfamily, 
  keywordstyle=\color{black}\bfseries, tabsize=4,
  stringstyle=\ttfamily, commentstyle=\it, extendedchars=true}
\usepackage{hyperref}
\usepackage{psfrag}
\usepackage{leftidx}

\newif\ifpdf
\ifx\pdfoutput\undefined
\pdffalse % we are not running PDFLaTeX
\else
\pdfoutput=1 % we are running PDFLaTeX
\pdftrue
\fi

\ifpdf
\usepackage[pdftex]{graphicx}
\else
\usepackage{graphicx}
\fi

\ifpdf
\DeclareGraphicsExtensions{.pdf, .jpg, .tif}
\else
\DeclareGraphicsExtensions{.eps, .jpg}
\fi

\newcommand{\C}{\mathbb{C}}
\newcommand{\R}{\mathbb{R}}
\newcommand{\N}{\mathbb{N}}
\newcommand{\Z}{\mathbb{Z}}
\newcommand{\Q}{\mathbb{Q}}
\newcommand{\Dune}{{\sf\bfseries DUNE}}

%The theorems
\theorembodyfont{\upshape}
\theoremheaderfont{\sffamily\bfseries}
\newtheorem{exc}{Excercise}[chapter]
\newtheorem{rem}[exc]{Remark}
\newtheorem{lst}{Listing}

\pagestyle{scrheadings}

\title{The Distributed and Unified Numerics Environment (DUNE) Grid
  Interface HOWTO}
\author{Peter Bastian}
\date{\today}

\begin{document}

\maketitle

\tableofcontents
\cleardoublepage

\chapter{Introduction}

\section{What is DUNE anyway?}

\section{Download and installation}

get dune

get dune-tutorial

and install them both

The purpose of this document is to get You started with the
``Distributed and Unified Numerics Environment'', in short DUNE. The
main web site for DUNE can be found at
%
\begin{center}
\href{http://hal.iwr.uni-heidelberg.de/dune/index.html}%
{\texttt{http://hal.iwr.uni-heidelberg.de/dune/index.html}}
\end{center}
%
We try to copy as little information as possible to avoid
inconsistencies. So for the installation of the DUNE software and
further packages that may be needed you should look at
\begin{center}
\href{http://hal.iwr.uni-heidelberg.de/dune/doc/installation-notes.html}%
{\texttt{http://hal.iwr.uni-heidelberg.de/dune/doc/installation-notes.html}}
\end{center}

\section{Code documentation}

Documentation of the files and classes in DUNE is provided in code and
can be extracted using the
doxygen\footnote{\href{http://www.stack.nl/~dimitri/doxygen/}{\texttt{http://www.stack.nl/$\sim$dimitri/doxygen/}}}
software available elsewhere. The in code documentation can either be built
on your machine or its latest version is available at
\begin{center}
\href{http://hal.iwr.uni-heidelberg.de/dune/doc/}%
{\texttt{http://hal.iwr.uni-heidelberg.de/dune/doc/}}
\end{center}

\section{How to start a new DUNE project}

\section{Licence}

\section{Contributing to DUNE}



\chapter{Getting started}

In this section we will take a quick tour through the abstract
grid interface provided by \Dune. This should give you an overview of
the different classes before we go into the details.

\section{Creating your first grid}

Let us start out with a replacement of the famous ``hello world''
program. If the listing is from a file that comes with this tutorial
its name is printed in the frame before the listing.

\begin{lst}[dune-grid-howto/gettingstarted.cc] \mbox{}

\lstinputlisting[basicstyle=\ttfamily\scriptsize,numbers=left, 
numberstyle=\tiny, numbersep=5pt]{../gettingstarted.cc}
\end{lst}

This program is quite simple. It starts with some includes in lines
4-6. The file \lstinline!config.h! has been produced by the
\lstinline!configure! script in the application's build system. It contains the
current configuration and can be used to compile different versions of
your code depending on the configuration selected. It is important
that this file is include before any other \Dune\ header files. The
next file \lstinline!dune/grid/sgrid.hh! includes the headers for the
\lstinline!SGrid! class which provides a special implementation of the
\Dune\ grid interface with an equidistant structured mesh in a cube in
any space dimension. Then \lstinline!dune/grid/common/gridinfo.hh!
loads the headers of some functions which print useful information
about a grid.

Since the dimension will be used as a template parameter in many
places below we define it as a constant in line number 11.
The \lstinline!SGrid! class template takes two template
parameters which are the dimensionality of the grid (its dimension)
and the dimension of the space where the grid is embedded (its world
dimension). The \lstinline!SGrid! class does only support the case
where dimension and world dimension are equal. For easy of writing we
define in line 12 the type \lstinline!GridType! using the selected value for the dimension.
Lines 13-15 prepare the arguments for the construction of an
\lstinline!SGrid! object. These arguments use the class template
\lstinline!FieldVector<T,n>! which is a vector with \lstinline!n!
components of type \lstinline!T!. You can either assign the same value
to all components in the constructor (as is done here) or you could
use \lstinline!operator[]! to assign values to individual components.
The variable \lstinline!N! defines the number of cells or elements to
be used in the respective dimension of the grid. \lstinline!L! defines
the coordinates of the lower left corner of the cube and \lstinline!H!
defines the extend of the cube in each space dimension. Finally in
line 16 we are now able to instantiate the \lstinline!SGrid!
object.

The only thing we do with the grid in this little example is printing
some information about it. After successfully running the executable
\lstinline!gettingstarted! you should see an output like this:

\begin{lstlisting}[basicstyle=\ttfamily\scriptsize]
=> SGrid(dim=3,dimworld=3)
level 0 codim[0]=27 codim[1]=108 codim[2]=144 codim[3]=64
leaf    codim[0]=27 codim[1]=108 codim[2]=144 codim[3]=64
leaf dim=3 geomTypes=((cube,3)[0]=27,(cube,2)[1]=108,(cube,1)[2]=144,(cube,0)[3]=64)
\end{lstlisting}

The first line tells you that you are looking at an \lstinline!SGrid!
object of the given dimensions. The \Dune\ grid interface supports
unstructured, locally refined, logically nested grids. The coarsest
grid is called level-0-grid or macro grid. Elements can be
individually refined into a number of smaller elements. Each element
of the macro grid and all its descendents obtained from refinement
form a tree structure. All elements at depth $n$ of a refinement tree
form the level-$n$-grid. All elements which are leafs of a refinement
tree together form the so-called leaf grid. The second line of the
output tells us that this grid object consists only of a single level
(level $0$) while the next line tells us that that level 0 coincides
also with the leaf grid in this case. Each line reports about the
number of grid entities which make up the grid. We see that there are
27 elements (codimension 0), 108 faces (codimension 1), 144 edges
(codimension 2) and 64 vertices (codimension 3) in the grid. The last
line reports on the different types of entities making up the grid. In
this case all entities are of type ``cube''.

\begin{exc} Try to play around with different grid sizes by assigning
  different values to the \lstinline!N! parameter. You can also change
  the dimension of the grid by varying \lstinline!dim!. Don't be
  modest. Also try dimensions 4 and 5!
\end{exc}

\section{Traversing a grid --- A first look at the grid interface}

After looking at very first simple example we are now ready to go on
to a more complicated example. Here it is:

\begin{lst}[dune-tutorial/traversal.cc] \mbox{}

\lstinputlisting[basicstyle=\ttfamily\scriptsize,numbers=left, 
numberstyle=\tiny, numbersep=5pt]{../traversal.cc}
\end{lst}

The \lstinline!main! function is pretty similar to previous one except
that we use a 2d grid that is refined once in line 115. This means
that we have a grid with two levels. The main work is done in a
call to the function \lstinline!traversal! in line 118. This function
is given in lines 13-101.

The function \lstinline!traversal! is a function template that is
parameterized by a class \lstinline!G! that implements the \Dune\ grid
interface. Thus it works on \textit{any} grid available in \Dune\
without any changes. 

The algorithm should work in any dimension so we extract the grid's
dimension and world dimension in lines 16 and 17. Next, each \Dune\
grid defines a type that it uses to represent positions. This type is
extracted in line 21 for ease of writing. 

A grid is considered to be a container of ``entities'' which are
abstractions for geometric objects like vertices, edges,
quadrilaterals, tetrahedra, and so on. This is very similar to the
standard template library (STL) which is part of any C++ system. A key
difference is, however, that there is not just one type of entity but
several. As in the STL the elements of any container can be accessed
with iterators which are generalized pointers. Again, a \Dune\ grid
knows several different iterators which provide access to the
different kinds of entities and which also provide different patterns
of access. 


\begin{lstlisting}[basicstyle=\ttfamily\scriptsize]
*** Traverse codim 0 leafs
visiting leaf (cube, 2) with first vertex at -1 -1
visiting leaf (cube, 2) with first vertex at 0 -1
visiting leaf (cube, 2) with first vertex at -1 0
visiting leaf (cube, 2) with first vertex at 0 0
there are/is 4 leaf element(s)

*** Traverse codim 0 level-wise
visiting (cube, 2) with first vertex at -1 -1
there are/is 1 element(s) on level 0

visiting (cube, 2) with first vertex at -1 -1
visiting (cube, 2) with first vertex at 0 -1
visiting (cube, 2) with first vertex at -1 0
visiting (cube, 2) with first vertex at 0 0
there are/is 4 element(s) on level 1


*** Traverse codim dim level-wise
visiting (cube, 0) at -1 -1
visiting (cube, 0) at 1 -1
visiting (cube, 0) at -1 1
visiting (cube, 0) at 1 1
there are/is 4 vertices(s) on level 0

visiting (cube, 0) at -1 -1
visiting (cube, 0) at 0 -1
visiting (cube, 0) at 1 -1
visiting (cube, 0) at -1 0
visiting (cube, 0) at 0 0
visiting (cube, 0) at 1 0
visiting (cube, 0) at -1 1
visiting (cube, 0) at 0 1
visiting (cube, 0) at 1 1
there are/is 9 vertices(s) on level 1
\end{lstlisting}

\begin{rem} Define the end iterator for efficiency. 
\end{rem}

\chapter{Main classes in the DUNE grid interface}

\section{Grid}

\section{Iterators}

\section{Entity}

\section{Geometry}



\chapter{Using different grids}

\section{Yasp}

\section{OneD}

\section{Alberta}

\section{UG}

\section{Alu3d}

\section{Using configuration information provided by configure}



\chapter{Reference elements}

\chapter{Quadrature rules}

\begin{lst}[dune-grid-howto/integrate.hh] \mbox{}

\lstinputlisting[basicstyle=\ttfamily\scriptsize,numbers=left, 
numberstyle=\tiny, numbersep=5pt]{../integrate.hh}
\end{lst}

\begin{lst}[dune-grid-howto/integration.cc] \mbox{}

\lstinputlisting[basicstyle=\ttfamily\scriptsize,numbers=left, 
numberstyle=\tiny, numbersep=5pt]{../integration.cc}
\end{lst}


\chapter{Attaching user data to a grid}

\chapter{Adaptivity}

\chapter{Parallelism}

\chapter{Input and output}





\bibliographystyle{plain}
\bibliography{grid-howto.bib}

\end{document}
