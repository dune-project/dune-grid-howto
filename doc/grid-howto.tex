\documentclass[11pt,a4paper,headinclude,footinclude,DIV16,normalheadings]{scrreprt}
\usepackage[automark]{scrpage2}
\usepackage[ansinew]{inputenc}
%\usepackage{german}
%\usepackage{bibgerm}
\usepackage{amsmath}
\usepackage{amsfonts}
\usepackage{theorem}
\usepackage{color}
\usepackage{listings}
\lstset{language=C++, basicstyle=\ttfamily, 
  keywordstyle=\color{black}\bfseries, tabsize=4,
  stringstyle=\ttfamily, commentstyle=\it, extendedchars=true}
\usepackage{hyperref}
\usepackage{psfrag}
\usepackage{makeidx}

\newif\ifpdf
\ifx\pdfoutput\undefined
\pdffalse % we are not running PDFLaTeX
\else
\pdfoutput=1 % we are running PDFLaTeX
\pdftrue
\fi

\ifpdf
\usepackage[pdftex]{graphicx}
\else
\usepackage{graphicx}
\fi

\ifpdf
\DeclareGraphicsExtensions{.pdf, .jpg, .tif}
\else
\DeclareGraphicsExtensions{.eps, .jpg}
\fi

\newcommand{\C}{\mathbb{C}}
\newcommand{\R}{\mathbb{R}}
\newcommand{\N}{\mathbb{N}}
\newcommand{\Z}{\mathbb{Z}}
\newcommand{\Q}{\mathbb{Q}}
\newcommand{\Dune}{{\sf\bfseries DUNE}}

%The theorems
\theorembodyfont{\upshape}
\theoremheaderfont{\sffamily\bfseries}
\newtheorem{exc}{Excercise}[chapter]
\newtheorem{rem}[exc]{Remark}
\newtheorem{lst}{Listing}

\pagestyle{scrheadings}

\title{The Distributed and Unified Numerics Environment (DUNE) Grid
  Interface HOWTO}

\author{Peter Bastian$^\ast$ \and 
Markus Blatt$^\ast$ \and
Andreas Dedner$^\dagger$ \and 
Christian Engwer$^\ast$ \and  
Robert Kl�fkorn$^\dagger$ \and 
Mario Ohlberger$^\dagger$ \and  
Oliver Sander$^\ddagger$}

\date{\today}

\publishers{%
\bigskip
{\normalsize $^\ast$Interdisziplin�res Zentrum f�r Wissenschaftliches Rechnen,
Universit�t Heidelberg,\\
Im Neuenheimer Feld 368, D-69120 Heidelberg, Germany}\\
%
\bigskip
{\normalsize $^\dagger$Abteilung f�r Angewandte Mathematik, Universit�t Freiburg,\\
Hermann-Herder-Str.~10, D-79104 Freiburg, Germany}\\
%
\bigskip
{\normalsize $^\ddagger$Institut f�r Mathematik II,\\ Freie Universit�t Berlin,
Arnimallee 2-6, D-14195 Berlin, Germany}\\
%
\bigskip
{\normalsize \texttt{http://hal.iwr.uni-heidelberg.de/dune/index.html}}\\
}

\makeindex

\begin{document}

\maketitle

\begin{abstract}
This document gives an introduction to the Distributed and Unified
Numerics Environment (\Dune). \Dune\ is a template library for the
numerical solution of partial differential equations. It is based on
the follwing principles: i) Seperation of data structures and
algorithms by abstract interfaces, ii) Efficient implementation of these
interfaces using generic programming techniques (templates) in C++ and
iii) Reuse of existing finite element packages with a large body of
functionality. This introduction covers only the abstract grid interface
of \Dune\ which is currently the most developed part. However, part of
\Dune\ are also the Iterative Solver Template Library (ISTL, providing a
large variety of solvers for sparse linear systems) and a flexible class
hierarchy for finite element methods. These will be described in
subsequent documents. Now have fun!
\end{abstract}

\tableofcontents


%%%%%%%%%%%%%%%%%%%%%%%%%%%%%%%%%%%%%%%%%%%%%%%%%%%%%%%%%%%%%%%%%%%%%%%%%%%
%%%%%%%%%%%%%%%%%%%%%%%%%%%%%%%%%%%%%%%%%%%%%%%%%%%%%%%%%%%%%%%%%%%%%%%%%%%
\chapter{Introduction}
%%%%%%%%%%%%%%%%%%%%%%%%%%%%%%%%%%%%%%%%%%%%%%%%%%%%%%%%%%%%%%%%%%%%%%%%%%%
%%%%%%%%%%%%%%%%%%%%%%%%%%%%%%%%%%%%%%%%%%%%%%%%%%%%%%%%%%%%%%%%%%%%%%%%%%%

\section{What is Dune anyway?}

\Dune\ is a software framework for the numerical solution of partial
differential equations with grid-based methods. It is based on the
following main principles:
\begin{itemize}
\item \textit{Seperation of data structures and
algorithms by abstract interfaces.} This provides more functionality
with less code and also ensures maintainability and
extendability of the framework.
\item \textit{Efficient implementation of these
interfaces using generic programming techniques}. Static polymorphism
allows the compiler to do more optimizations, in particular function
inlining, which in turn allows the interface to have very small
functions (implemented by one or few machine instructions) without a
severe performance penalty. In essence the algorithms are parametrized
with a particular data structure and the interface is removed at
compile time. Thus the resulting code is as efficient as if it would
have been written for the special case.
\item \textit{Reuse of existing finite element packages with a large body of
functionality.} In particular the finite element codes UG, \cite{ug},
Alberta, \cite{Alberta}, and ALU3d, \cite{ALU3d}, have been
adapted to the \Dune\ framework. Thus, parallel and adaptive meshes with
multiple element types and refinement rules are available. All these
packages can be linked together in one executable.
\end{itemize}

The framework consists of a number of modules which are different
states of maturity. In particular these are:
\begin{itemize}
\item \textit{Grid interface.} This is the most mature module that is
  covered in this document. It defines nonconforming, hierarchically
  nested, multi-element-type, parallel grids in arbitrary space dimensions. 
\item \textit{Iterative Solver Template Library.} Provides generic
  sparse matrix/vector classes and a variety of solvers based on these
  classes. A special feature is the use of templates to exploit the
  recursive block structure of finite element matrices at compile
  time. Available solvers include Krylov methods, (block-) incomplete
  decompositions and aggregation-based algebraic multigrid.
\item \textit{Freiburg Finite Element Hierarchy.} A flexible class
  hierarchy for finite elements. Explicit cell-centered finite volume
  and discontinuous Galerkin methods for hyperbolic problems,
  e.~g.~transport in porous media and inviscid fluid flow have been implemented.
\item \textit{Heidelberg Finite Element Hierarchy.} Another flexible
  class hierarchy for finite elements. Standard and discontinuous
  Galerkin finite elements for elliptic problems, e.~g.~Laplacian,
  linear elasticity and Stokes have been implemented.
\item \textit{Input/Output.} Graphical output with several packages is
  available, e.~g.~file output to IBM data explorer and VTK (parallel
  XML format for unstructured grids). The graphics package Grape,
  \cite{Grape} has been integrated in interactive mode.
\end{itemize}

Before starting to work with \Dune\ you might want to update your
knowledge about C++ and templates in particular. For that you should
have the bible, \cite{Stroustrup}, at your desk. A good introduction,
besides its age, is still the book by Barton and Nackman,
\cite{BN}. The definitive guide to template programming is
\cite{VandervoordeJosuttis}. A very useful compilation of template
programming tricks with application to scientific computing is given
in \cite{Veldhui99} (if you can't find it on the web contact us).

\section{Download}

\Dune\ and its applications are distributed under the GNU Lesser
General Public License 
Version 2.1\footnote{\href{http://www.gnu.org/licenses/lgpl.html}%
{\texttt{http://www.gnu.org/licenses/lgpl.html}}}.


\minisec{Dune}

The source code of the \Dune\ framework can be
downloaded from the web page (follow the instructions given there)
%
\begin{center}
\href{http://hal.iwr.uni-heidelberg.de/dune/download.html}%
{\texttt{http://hal.iwr.uni-heidelberg.de/dune/download.html}}
\end{center}
%

\minisec{Dune grid HOWTO}

With the \Dune\ library itself you cannot do much. You need an
application that uses \Dune\ to do something useful. One such
application is the \Dune\ grid HOWTO which contains the examples
described in this
document. It can be downloaded from the same web page as the \Dune\
library.

\section{Installation}

The official installation instructions are available on the web page
%
\begin{center}
\href{http://hal.iwr.uni-heidelberg.de/dune/doc/installation-notes.html}%
{\texttt{http://hal.iwr.uni-heidelberg.de/dune/doc/installation-notes.html}}
\end{center}

Obviously we do not want to copy all this information because it might
get outdated and inconsistent then. To make this document
self-contained we describe only how to install a vanilla version without
any additional packages. Moreover, we assume that you use a UNIX
system. If you have the Redmont system then ask them how to install it.

\minisec{Required software}

In order to build the \Dune\ framework the following
software must be available on your machine:
\begin{itemize}
\item \lstinline!automake! in version $\geq 1.5$.
\item \lstinline!autoconf! in version $\geq 2.50$.
\item \lstinline!libtool!.
\item \lstinline!g++! (the GNU C++ compiler) in version $\geq 3.4.1$ or any
  other C++-compiler that is able to compile it. E.~g.~the INTEL
  compiler works as well.
\end{itemize}


\minisec{From official version download} 

\noindent Add this section later\ldots

\minisec{From tarballs downloaded via view CVS} 

So you have downloaded the two files
\lstinline!dune.tar.gz! and \lstinline!dune-grid-howto.tar.gz!. Put
them in a directory of your choice extract the archives

\begin{lstlisting}[basicstyle=\ttfamily\scriptsize]
> cd <your directory>
> tar zxvf dune.tar.gz
> tar zxvf dune-grid-howto.tar.gz
\end{lstlisting}

Now you can join the folks who have downloaded the CVS repositories.

\minisec{From CVS download} 

You have checked out out two directories \lstinline!dune! and 
\lstinline!dune-grid-howto! from the CVS server. If you have not done
that already put them in a directory of your choice next to each other.

\noindent First we configure and build \Dune:

\begin{lstlisting}[basicstyle=\ttfamily\scriptsize]
> cd <your directory>/dune
> ./autogen.sh
> ./configure CXXFLAGS="-g -O0" CFLAGS="-g -O0" CXX="g++-4.0" CC="gcc-4.0" --enable-dunedevel
> make
\end{lstlisting}
This configures and builds \Dune\ with debugging flags using version
$4.0$ of the GNU C++ and C compiler. You may supply your own compiler
with your favourite options.

\noindent Now we configure and build the \Dune grid HOWTO:

\begin{lstlisting}[basicstyle=\ttfamily\scriptsize]
> cd <your directory>/dune-grid-howto
> ./autogen.sh ../dune
> ./configure --with-dune=../dune CXXFLAGS="-g -O0" CFLAGS="-g -O0" \
  CXX="g++-4.0" CC="gcc-4.0" --enable-dunedevel
> make gettingstarted
> make traversal
> make integration
\end{lstlisting}

For the remaining targets using adaptive grids you need to install one
ore more of the additional libraries UG, Alberta and ALU3d.

The \lstinline!./configure! script prints a list of add-on packages it
has recognized. The output may look like this:

\begin{lstlisting}[basicstyle=\ttfamily\scriptsize]
Alberta..........: no
ALUGrid..........: no
AmiraMesh........: no
BLAS-lib.........: no
Grape............: no
MPI..............: MPICH
METIS............: no
ParMETIS.........: no
OpenGL...........: yes
UG...............: no
\end{lstlisting}

Here, the OpenGL library and and MPI (message passing interface)
library have been found.


\section{Code documentation}

Documentation of the files and classes in DUNE is provided in code and
can be extracted using the
doxygen\footnote{\href{http://www.stack.nl/~dimitri/doxygen/}{\texttt{http://www.stack.nl/$\sim$dimitri/doxygen/}}}
software available elsewhere. The in code documentation can either be built
on your machine or its latest version is available at
\begin{center}
\href{http://hal.iwr.uni-heidelberg.de/dune/doc/}%
{\texttt{http://hal.iwr.uni-heidelberg.de/dune/doc/}}
\end{center}

\section{How to start a new DUNE project}

\section{Licence}

\Dune\ is distributed under the GNU Lesser General Public License
Version 2.1\footnote{\href{http://www.gnu.org/licenses/lgpl.html}%
{\texttt{http://www.gnu.org/licenses/lgpl.html}}}.

\section{Contributing to DUNE}




%%%%%%%%%%%%%%%%%%%%%%%%%%%%%%%%%%%%%%%%%%%%%%%%%%%%%%%%%%%%%%%%%%%%%%%%%%%
%%%%%%%%%%%%%%%%%%%%%%%%%%%%%%%%%%%%%%%%%%%%%%%%%%%%%%%%%%%%%%%%%%%%%%%%%%%
\chapter{Getting started}
%%%%%%%%%%%%%%%%%%%%%%%%%%%%%%%%%%%%%%%%%%%%%%%%%%%%%%%%%%%%%%%%%%%%%%%%%%%
%%%%%%%%%%%%%%%%%%%%%%%%%%%%%%%%%%%%%%%%%%%%%%%%%%%%%%%%%%%%%%%%%%%%%%%%%%%

In this section we will take a quick tour through the abstract
grid interface provided by \Dune. This should give you an overview of
the different classes before we go into the details.

\section{Creating your first grid}

Let us start with a replacement of the famous ``hello world''
program given below.

\begin{lst}[File dune-grid-howto/gettingstarted.cc] \mbox{}

\lstinputlisting[basicstyle=\ttfamily\scriptsize,numbers=left, 
numberstyle=\tiny, numbersep=5pt]{../gettingstarted.cc}
\end{lst}

This program is quite simple. It starts with some includes in lines
4-6. The file \lstinline!config.h! has been produced by the
\lstinline!configure! script in the application's build system. It contains the
current configuration and can be used to compile different versions of
your code depending on the configuration selected. It is important
that this file is include before any other \Dune\ header files. The
next file \lstinline!dune/grid/sgrid.hh! includes the headers for the
\lstinline!SGrid! class which provides a special implementation of the
\Dune\ grid interface with an equidistant structured mesh in a cube in
any space dimension. Then \lstinline!dune/grid/common/gridinfo.hh!
loads the headers of some functions which print useful information
about a grid.

Since the dimension will be used as a template parameter in many
places below we define it as a constant in line number 11.
The \lstinline!SGrid! class template takes two template
parameters which are the dimensionality of the grid (its dimension)
and the dimension of the space where the grid is embedded (its world
dimension). The \lstinline!SGrid! class does only support the case
where dimension and world dimension are equal. For easy of writing we
define in line 12 the type \lstinline!GridType! using the selected
value for the dimension. All identifiers of the \Dune\
framework are within the \lstinline!Dune! namespace.

Lines 13-15 prepare the arguments for the construction of an
\lstinline!SGrid! object. These arguments use the class template
\lstinline!FieldVector<T,n>! which is a vector with \lstinline!n!
components of type \lstinline!T!. You can either assign the same value
to all components in the constructor (as is done here) or you could
use \lstinline!operator[]! to assign values to individual components.
The variable \lstinline!N! defines the number of cells or elements to
be used in the respective dimension of the grid. \lstinline!L! defines
the coordinates of the lower left corner of the cube and \lstinline!H!
defines the extend of the cube in each space dimension. Finally in
line 16 we are now able to instantiate the \lstinline!SGrid!
object.

The only thing we do with the grid in this little example is printing
some information about it. After successfully running the executable
\lstinline!gettingstarted! you should see an output like this:

\begin{lst}[Output of gettingstarted] \mbox{}

\begin{lstlisting}[basicstyle=\ttfamily\scriptsize]
=> SGrid(dim=3,dimworld=3)
level 0 codim[0]=27 codim[1]=108 codim[2]=144 codim[3]=64
leaf    codim[0]=27 codim[1]=108 codim[2]=144 codim[3]=64
leaf dim=3 geomTypes=((cube,3)[0]=27,(cube,2)[1]=108,(cube,1)[2]=144,(cube,0)[3]=64)
\end{lstlisting}
\end{lst}

The first line tells you that you are looking at an \lstinline!SGrid!
object of the given dimensions. The \Dune\ grid interface supports
unstructured, locally refined, logically nested grids. The coarsest
grid is called level-0-grid or macro grid. Elements can be
individually refined into a number of smaller elements. Each element
of the macro grid and all its descendents obtained from refinement
form a tree structure. All elements at depth $n$ of a refinement tree
form the level-$n$-grid. All elements which are leafs of a refinement
tree together form the so-called leaf grid. The second line of the
output tells us that this grid object consists only of a single level
(level $0$) while the next line tells us that that level 0 coincides
also with the leaf grid in this case. Each line reports about the
number of grid entities which make up the grid. We see that there are
27 elements (codimension 0), 108 faces (codimension 1), 144 edges
(codimension 2) and 64 vertices (codimension 3) in the grid. The last
line reports on the different types of entities making up the grid. In
this case all entities are of type ``cube''.

\begin{exc} Try to play around with different grid sizes by assigning
  different values to the \lstinline!N! parameter. You can also change
  the dimension of the grid by varying \lstinline!dim!. Don't be
  modest. Also try dimensions 4 and 5!
\end{exc}

\section{Traversing a grid --- A first look at the grid interface}

After looking at very first simple example we are now ready to go on
to a more complicated one. Here it is:

\begin{lst}[File dune-grid-howto/traversal.cc] \mbox{}
\nopagebreak
\lstinputlisting[basicstyle=\ttfamily\scriptsize,numbers=left, 
numberstyle=\tiny, numbersep=5pt]{../traversal.cc}
\end{lst}

The \lstinline!main! function near the end of the listing
is pretty similar to previous one except
that we use a 2d grid for the unit square that just consists of one
cell. In line 106 this cell is refined once using the standard method
of grid refinement of the implementation. Here, the cell is refined
into four smaller cells. The main work is done in a
call to the function \lstinline!traversal! in line 109. 
This function is given in lines 12-92.

The function \lstinline!traversal! is a function template that is
parameterized by a class \lstinline!G! that is assumed to
implement the \Dune\ grid interface. 
Thus, it will work on \textit{any} grid available in \Dune\
without any changes. We now go into the details of this function.

The algorithm should work in any dimension so we extract the grid's
dimension in line 16. Next, each \Dune\
grid defines a type that it uses to represent positions. This type is
extracted in line 20 for later use. 

A grid is considered to be a container of ``entities'' which are
abstractions for geometric objects like vertices, edges,
quadrilaterals, tetrahedra, and so on. This is very similar to the
standard template library (STL), see e.~g.~\cite{Stroustrup},
which is part of any C++ system. 
A key difference is, however, that there is not just one type of entity but
several. As in the STL the elements of any container can be accessed
with iterators which are generalized pointers. Again, a \Dune\ grid
knows several different iterators which provide access to the
different kinds of entities and which also provide different patterns
of access. 

Line 29 extracts the type of an iterator from the grid
class. \lstinline!Codim! is a \lstinline!struct! within the grid class
that takes an integer template parameter specifying the codimension
over which to iterate. Within the \lstinline!Codim! structure the type
\lstinline!LeafIterator! is defined. Since we specified codimension 0
this iterator is used to iterate
over the elements which are not refined any further, i.~e.~which are
the leaves of the refinement trees.

The \lstinline!for!-loop in lines 33-34 now visits every such
element. The \lstinline!leafbegin! and \lstinline!leafend! on the grid
class deliver the first leaf element and one past the last leaf
element. Note that the \lstinline!template! keyword must be used and
template parameters are passed explicitely. Within the loop body in
lines 35-41 the iterator \lstinline!it! acts like a pointer to an entity of
dimension \lstinline!dim! and codimension 0. The exact type would be
\lstinline!typename G::template Codim<0>::Entity! just to mention
it.

An important part of an entity is its geometrical shape and
position. All geometrical information is factored out into a
sub-object that can be accessed via the \lstinline!geometry()!
method. The geometry object is in general a mapping from a $d$-dimensional
polyhedral reference element to $w$ dimensional space. Here we have
$d=$ \lstinline!G::dimension! and $w=$
\lstinline!G::dimensionworld!. This mapping is also called the ``local to
global'' mapping.
The corresponding reference element has a certain type which is
extracted in line 36. Since the reference elements are polyhedra they
consist of a finite number of corners. The images of the corners under
the local to global map can be accessed via an
\lstinline!operator[]!. Lines 37-39 print the geometry type and the
position of the first corner of the element. Then line 40 just counts
the number of elements visited.

Suppose now that we wanted to iterate over the vertices of the leaf
grid instead of the elements. Now vertices have the codimension
\lstinline!dim! in a \lstinline!dim!-dimensional grid and a
corresponding iterator is provided by each grid class. It is extracted
in line 51 for later use. The \lstinline!for!-loop starting in line 55
is very similar to the first one except that it now uses the
\lstinline!VertexLeafIterator!.  
As you can see the different entities can be accessed with the same
methods. We will see later that codimensions 0 and \lstinline!dim! are
specializations with an extended interface compared to all other
codimensions. You can also access the codimensions between 0 and
\lstinline!dim!. However, currently not all implementations of the
grid interface support these intermediate codimensions (though this
does not restrict the implementation of finite element methods with
degrees of freedom associated to, say, faces).

Finally, we show in lines 73-91 how the hierarchic structure of the
mesh can be accessed. To that end a \lstinline!LevelIterator! is
used. It provides access to all entities of a given codimension (here
0) on a given grid level. The coarsest grid level (the initial macro
grid) has number zero and the number of the finest grid level is
returned by the \lstinline!maxLevel()! method of the grid. 
The methods \lstinline!lbegin()! and \lstinline!lend()! on the grid
deliver iterators to the first and one-past-the-last entity of a given
grid level supplied as an integer argument to these methods. 

The following listing shows the output of the program.

\begin{lst}[Output of traversal] \mbox{}

\begin{lstlisting}[basicstyle=\ttfamily\scriptsize]
*** Traverse codim 0 leaves
visiting leaf (cube, 2) with first vertex at -1 -1
visiting leaf (cube, 2) with first vertex at 0 -1
visiting leaf (cube, 2) with first vertex at -1 0
visiting leaf (cube, 2) with first vertex at 0 0
there are/is 4 leaf element(s)

*** Traverse codim 2 leaves
visiting (cube, 0) at -1 -1
visiting (cube, 0) at 0 -1
visiting (cube, 0) at 1 -1
visiting (cube, 0) at -1 0
visiting (cube, 0) at 0 0
visiting (cube, 0) at 1 0
visiting (cube, 0) at -1 1
visiting (cube, 0) at 0 1
visiting (cube, 0) at 1 1
there are/is 9 leaf vertices(s)

*** Traverse codim 0 level-wise
visiting (cube, 2) with first vertex at -1 -1
there are/is 1 element(s) on level 0

visiting (cube, 2) with first vertex at -1 -1
visiting (cube, 2) with first vertex at 0 -1
visiting (cube, 2) with first vertex at -1 0
visiting (cube, 2) with first vertex at 0 0
there are/is 4 element(s) on level 1
\end{lstlisting}
\end{lst}

\begin{rem} Define the end iterator for efficiency. 
\end{rem}

\begin{exc} Play with different dimensions, codimension
  (\lstinline!SGrid! supports all codimenions) and refinements.
\end{exc}

\begin{exc} The method \lstinline!corners()! of the geometry returns
  the number of corners of an entity. Modify the code such that the
  positions of all corners are printed.
\end{exc}


%%%%%%%%%%%%%%%%%%%%%%%%%%%%%%%%%%%%%%%%%%%%%%%%%%%%%%%%%%%%%%%%%%%%%%%%%%%
%%%%%%%%%%%%%%%%%%%%%%%%%%%%%%%%%%%%%%%%%%%%%%%%%%%%%%%%%%%%%%%%%%%%%%%%%%%
\chapter{The DUNE grid interface}
%%%%%%%%%%%%%%%%%%%%%%%%%%%%%%%%%%%%%%%%%%%%%%%%%%%%%%%%%%%%%%%%%%%%%%%%%%%
%%%%%%%%%%%%%%%%%%%%%%%%%%%%%%%%%%%%%%%%%%%%%%%%%%%%%%%%%%%%%%%%%%%%%%%%%%%


\section{Grid definition}

There is a great variety of grids: conforming and non-conforming
grids, single-element-type and multiple-element-type grids, locally
and globally refined grids, nested and non-nested grids,
bisection-type grids, red-green-type grids, sparse grids and so on. In
this section we describe in some detail the type of grids that are
covered by the \Dune\ grid interface.

\minisec{Reference elements}

A computational grid is a nonoverlapping subdivision of a domain
$\Omega\subset\R^w$ into elements of ``simple'' shape. Here ``simple''
means that the element can be represented as the image of a reference
element\index{reference element} under a transformation. A reference element is a convex
polytope, which is a bounded intersection of a finite set of
half-spaces. 

\minisec{Dimension and world dimension}

A grid has a dimension $d$ which is the dimensionality of
its reference elements. Clearly we have $d\leq w$. In the case $d<w$ the grid
discretizes a $d$-dimensional manifold. 

\minisec{Faces, entities and codimension}

The intersection of a $d$-dimensional convex polytope (in
$d$-dimensional space) with a
tangent plane is called a face (note that there are faces of
dimensionality $0,\ldots,d-1$). Consequently, a face of a grid element
is defined as the image of a face of its reference element under the
transformation. The elements and faces of elements of a grid are
called its entities. An entity is said to be of codimension $c$ if it
is a $d-c$-dimensional object. Thus the elements of the grid are
entities of codimension 0, facets of an element have codimension 1,
edges have codimension $d-1$ and vertices have codimension $d$.

\minisec{Conformity}

Computational grids come in a variety of flavours: A
{conforming} grid is one where the intersection of two
elements is either empty or a face of each of the two elements. 
Grids where the intersection of two elements may have an
arbitrary shape are called {nonconforming}. 

\minisec{Element types}

A {simplicial} grid is one where the reference elements are
simplices. In a {multi-element-type} grid a finite number of
different reference elements are allowed. The \Dune\ grid interface
can represent conforming as well as non-conforming grids.

\minisec{Hierarchically nested grids, macro grid}

A {hierarchically nested} grid consists of a collection of $J+1$
grids that are subdivisions of nested domains $$\Omega=\Omega_0 \supseteq \Omega_1 \supseteq
\ldots \supseteq \Omega_J.$$ Note that only $\Omega_0$ is required to
be identical to $\Omega$. If $\Omega_0=\Omega_1=\ldots=\Omega_J$ the
grid is {globally refined}, otherwise it is {locally refined}.
The grid that discretizes $\Omega_0$ is called the macro grid and its
elements the macro elements. The
grid for $\Omega_{l+1}$ is obtained from the grid
for $\Omega_l$ by possibly subdividing each of its elements into
smaller elements. Thus, each element of the macro grid and the
elements that are obtained from refining it form a tree structure. The
grid discretizing $\Omega_l$ with $0\leq l \leq J$ is called the level-$l$-grid and its
elements are obtained from an $l$-fold refinement of some macro elements.

\minisec{Leaf grid}

Due to the nestedness of the domains we can partition the domain
$\Omega$ into $$\Omega = \Omega_J \cup \bigcup_{l=0}^{J-1}
\Omega_l\setminus\Omega_{l+1}.$$ As a consequence of the hierarchical
construction a computational grid discretizing $\Omega$ can be
obtained by taking the elements of the level-$J$-grid plus
the elements of the level-$J-1$-grid in the region
$\Omega_{J-1}\setminus\Omega_{J}$ plus the elements of the level-$J-2$-grid in the region
$\Omega_{J-2}\setminus\Omega_{J-1}$ and so on plus the elements of the level-$0$-grid in the region
$\Omega_{0}\setminus\Omega_{1}$. The grid resulting from this
procedure is called the leaf grid
because it is formed by the leaf elements of the trees emanating at
the macro elements. 

\minisec{Refinement rules}

There is a variety of ways how to hierarchically refine a grid. The
refinement is called conforming if the leaf grid is always a
conforming grid, otherwise the refinement is called
non-conforming. Note that the grid on each level $l$ might be
conforming while the leaf grid is not. 
There are also many ways how to subdivide an individual element into
smaller elements. Bisection always subdivides elements into two
smaller elements, thus the resulting data structure is a binary tree
(independent of the dimension of the grid). Bisection is sometimes
calles ``green'' refinement. The so-called ``red'' refinement is the
subdivision of an element into $2^d$ smaller elements, which is most
obvious for cube elements. In many practical situation anisotropic
refinement, i.~e.~refinement in a preferred direction, may be required.

\minisec{Summary}

The \Dune\ grid interface is able to represent grids with the
following properties:
\begin{itemize}
\item Arbitrary dimension.
\item Entities of all codimensions.
\item Any kind of reference elements (you could define the icosahedron
  as a reference element if you wish).
\item Conforming and non-conforming grids.
\item Grids are always hierarchically nested.
\item Any type of refinement rules.
\item Conforming and non-conforming refinement.
\item Parallel, distributed grids.
\end{itemize}



\section{Grid}


\section{Prerequisites}

The description of the interface assumes familarity with the concepts
of the Standard Template Library (STL) such as containers, iterators, value
types, etc. On the other hand, the \Dune\ grid interface is not fully
compatible with the STL, i.~e.~if a certain class is said to be of
forward iterator type this does not imply that it has exactly the same
members as a STL forward iterator.  

\subsection{Enumeration Types}

There are several enumeration types which are used for specifying a
finite number of choices for certain parameters in the interface.

\minisec{\tt GridIdentifier\index{{GridIdentifier}@{\tt
      GridIdentifier}}} This type contains one identifier for each
type implementing the \Dune\ grid interface. Currently these are
\begin{quote}
\raggedright
\lstinline!SGrid_Id!, \lstinline!AlbertaGrid_Id!, \lstinline!SimpleGrid_Id!, \lstinline!UGGrid_Id!, 
\lstinline!YaspGrid_Id!, \lstinline!ALU3dGrid_Id! and \lstinline!OneDGrid_Id!.
\end{quote}

\minisec{\tt AdaptionState\index{{AdaptionState}@{\tt AdaptionState}}}
The adaption state may take the values
\begin{quote}
\raggedright
\lstinline!NONE!, \lstinline!COARSEN! and \lstinline!REFINED!.
\end{quote}
It is used to indicate the state of a grid entity after refinement.

\minisec{\tt PartitionType\index{{PartitionType}@{\tt PartitionType}}}

In a parallel computation the grid is partitioned to a given number of
processes in an overlapping or non-overlapping fashion. Each entity of
a grid part in a process is assigned a value of
\lstinline!PartitionType!. It consist of the values
\begin{quote}
\raggedright
\lstinline!InteriorEntity!, \lstinline!BorderEntity!,
\lstinline!OverlapEntity!, \lstinline!FrontEntity! and
\lstinline!GhostEntity!.
\end{quote}


\minisec{\tt PartitionIteratorType\index{{PartitionIteratorType}@{\tt
      PartitionIteratorType}}}
with the values
\begin{quote}
\raggedright
\lstinline!Interior_Partition!,
\lstinline!InteriorBorder_Partition!,
\lstinline!Overlap_Partition!,
\lstinline!OverlapFront_Partition!,
\lstinline!All_Partition! and
\lstinline!Ghost_Partition!
\end{quote}
parametrizes the iterators for the entities of different partition
type.

\minisec{\tt InterfaceType\index{{InterfaceType}@{\tt InterfaceType}}}
with the values
\begin{quote}
\raggedright
\lstinline!InteriorBorder_InteriorBorder_Interface!,
\lstinline!InteriorBorder_All_Interface!,
\lstinline!Overlap_OverlapFront_Interface!,
\lstinline!Overlap_All_Interface! and
\lstinline!All_All_Interface!
\end{quote}
is used in parametrizing the communication between processes.


\minisec{\tt
  CommunicationDirection\index{{CommunicationDirection}@{\tt
      CommunicationDirection}}} 
with the values
\begin{quote}
\raggedright
\lstinline!ForwardCommunication! and
\lstinline!BackwardCommunication!
\end{quote}
is used in parametrizing the communication between processes.

\subsection{Notation}

\begin{description}
\item[{\tt G}:] A type that is a model of Grid.
\item[\texttt{d},\texttt{c}:] Integer constants denoting
  dimension and codimension.
\item[\texttt{p}:] A constant of type \lstinline!PartitionIteratorType!.
\end{description}

\subsection{Constants}


\minisec{\tt G::dimension\index{dimension@{\tt dimension}}}
The dimensionality $d$ of the elements of the grid. Note that all elements
of the grid are of the same dimension, i.~e.~it is not possible to
have elements of different dimension in one grid. Entities with a
dimension $d^\prime<d$ only occur as faces of a $d$-dimensional element.

\minisec{\tt G::dimensionworld\index{dimensionworld@{\tt dimensionworld}}}
The dimension $w\geq d$ of the domain $\Omega$ that is discretized by the grid.

\subsection{Associated Types}


\minisec{\tt G::ctype\index{ctype@{\tt ctype}}} 
The type used for representing coordinates in the grid.

\minisec{\tt G::Traits::Codim<c>::Entity\index{{entity}@{\tt Entity}}}
The type for the entity of codimension \lstinline!c!.

\minisec{\tt G::Traits::Codim<c>::Geometry\index{{geometry}@{\tt
      Geometry}}}
The type describing a map from a reference element of dimension
\lstinline!dimension-c! to a space of dimension \lstinline!dimensionworld!.

\minisec{\tt G::Traits::Codim<c>::LocalGeometry\index{{local geometry}@{\tt LocalGeometry}}}
The type describing a map from a reference element of dimension
\lstinline!dimension-c! to a a space of dimension \lstinline!dimension!.

\minisec{\tt G::Traits::Codim<c>::EntityPointer\index{{entity
      pointer}@{\tt EntityPointer}}}
A type that behaves as a const pointer to an entity of
codimension \lstinline!c!.

\minisec{\tt
  G::Traits::Codim<c>::Partition<p>::LevelIterator\index{{level
      iterator}@{\tt LevelIterator}}}  
A type of iterator that may be used to examine, but not to modify, the
entities of codimension \lstinline!c! with partition type
\lstinline!p! on a certain level of the grid, i.~e.~the increment of
the iterator adjusts it to the next entity on that level.


\minisec{\tt
  G::Traits::Codim<c>::Partition<p>::LeafIterator\index{{leaf
      iterator}@{\tt LeafIterator}}} 
A type of iterator that may be used to examine, but not to modify, the
entities of codimension \lstinline!c! with partition type
\lstinline!p! in the leaf grid, i.~e.~the increment of
the iterator adjusts it to the next entity in the leaf grid. 

\minisec{\tt G::Traits::IntersectionIterator\index{{intersection
      iterator}@{\tt IntersectionIterator}}}
A type of iterator that allows to examine, but not to modify, the
intersections of codimension 1 of an element (entity of codimension 0)
with other elements.

\minisec{\tt G::Traits::HierarchicIterator\index{{hierarchic
      iterator}@{\tt HierarchicIterator}}} 
A type of iterator that allows to examine, but not to modify, entities
of codimension 0 that result from refinement of an entity of
codimension 0.

\minisec{\tt G::Traits::Codim<c>::LevelIterator\index{{level
      iterator}@{\tt LevelIterator}}}
Shortcut for the \lstinline!LevelIterator! with partition type
\lstinline!All_Partition!. 

\minisec{\tt G::Traits::Codim<c>::LeafIterator\index{{leaf
      iterator}@{\tt LeafIterator}}} 
Shortcut for the \lstinline!LeafIterator! with partition type
\lstinline!All_Partition!. 


\minisec{\tt G::Traits::LevelIndexSet\index{{level index set}@{\tt
      LevelIndexSet}}}
A type that provides a consecutive, but non persistent, numbering for
entities on a grid level. 

\minisec{\tt G::Traits::LeafIndexSet\index{{leaf index set}@{\tt
      LeafIndexSet}}} 
A type that provides a consecutive, but non persistent, numbering for
entities in the leaf grid. 

\minisec{\tt G::Traits::GlobalIdSet\index{{global id set}@{\tt
      GlobalIdSet}}} 
A type that provides a unique and persistent numbering for
all entities in the grid. The numbering is unique over all processes
over which the grid is partitioned. The numbering is not necessarily
consecutive.  

\minisec{\tt G::Traits::LocalIdSet\index{{local id set}@{\tt
      LocalIdSet}}} 
A type that provides a unique and persistent numbering for
all entities in the grid. The numbering is only unique in a single process
and it is not necessarily consecutive.  



\subsection{Definitions}

\subsection{Valid Expressions}

\subsection{Expression semantics}

\subsection{Complexity guarantees}

\subsection{Invariants}

\subsection{Implementation details}

\subsection{Notes}


\section{Iterators}

\section{Entity}

\section{Geometry Type}

\section{Geometry}

\section{Collective communication}


\chapter{Using different grids}

\begin{lst}[File dune-grid-howto/unitcube.hh] \mbox{}
\nopagebreak
\lstinputlisting[basicstyle=\ttfamily\scriptsize,numbers=left, 
numberstyle=\tiny, numbersep=5pt]{../unitcube.hh}
\end{lst}


\section{Yasp}

\section{OneD}

\section{Alberta}

\section{UG}

\section{Alu3d}

\section{Using configuration information provided by configure}



\chapter{Reference elements}

\chapter{Quadrature rules}

\begin{lst}[dune-grid-howto/integrate\_domain.hh] \mbox{}

\lstinputlisting[basicstyle=\ttfamily\scriptsize,numbers=left, 
numberstyle=\tiny, numbersep=5pt]{../integrate_domain.hh}
\end{lst}

\begin{lst}[dune-grid-howto/integration.cc] \mbox{}

\lstinputlisting[basicstyle=\ttfamily\scriptsize,numbers=left, 
numberstyle=\tiny, numbersep=5pt]{../integration.cc}
\end{lst}


\chapter{Attaching user data to a grid}

\chapter{Adaptivity}

\begin{lst}[File dune-grid-howto/integrate\_entity.hh] \mbox{}
\nopagebreak
\lstinputlisting[basicstyle=\ttfamily\scriptsize,numbers=left, 
numberstyle=\tiny, numbersep=5pt]{../integrate_entity.hh}
\end{lst}

\begin{lst}[File dune-grid-howto/adaptiveintegration.cc] \mbox{}
\nopagebreak
\lstinputlisting[basicstyle=\ttfamily\scriptsize,numbers=left, 
numberstyle=\tiny, numbersep=5pt]{../adaptiveintegration.cc}
\end{lst}

\chapter{Parallelism}

\chapter{Input and output}





\bibliographystyle{plain}
\bibliography{grid-howto.bib}

\printindex

\end{document}
